%!TeX encoding = UTF-8


\documentclass[english, nochinese]{beamer}

\usepackage{amsfonts,amsmath}
\usetheme{Berkeley}

\title{Outline: SciPy}

\author{Alisomia}
\date{Updated on April 3, 2018}

\begin{document}


\begin{frame}
\titlepage
\end{frame}


\begin{frame}
\tableofcontents
\end{frame}


\section{What is SciPy}


\begin{frame}

\end{frame}


\section{Integrate}


\begin{frame}
\frametitle{scipy.integrate}

\begin{itemize}

\item quad(func, a, b, args=(), weight=None, wvar=None, wopts=None)
\item  dblquad(func, a, b, gfun, hfun, args=()) 
\item tplquad(func, a, b, gfun, hfun, qfun, rfun, args=())

\end{itemize}
About \textbf{dblquad} and \textbf{tplquad}, pay  attention to the position of args: e.g. func(z, y, x) from $x \in [a,b], y \in [gfun(x), hfun(x)], z \in [qfun(x, y), rfun(x, y)] $
\end{frame}

\begin{frame}
\frametitle{scipy.integrate}
\begin{itemize}
\item odeint(func, y0, t, args=())
\item ode(f, jac)
\end{itemize}
the input of \textbf{odeint} is func(y0, t), and jac in \textbf{ode(f,jac)} is df/dy\\
the difference between \textbf{ode} and \textbf{odeint} is \textbf{odeint} solves the ODE directly but \textbf{ode} returns a class
\end{frame}


\section{Optimization}
\begin{frame}
\frametitle{scipy.optimize}
\begin{itemize}
\item curve\_fit(f, x, y)
\item  fsolve(func, x0, args=(), fprime=None)
\item  fmin(func, x0, args=())
\item linprog({c, A\_ub=None, b\_ub=None, A\_eq=None, b\_eq=None)})
\end{itemize}
Minimize:     $c^T * x$

Subject to:  $ A_ub * x <= b_ub, A_eq * x == b_eq$

In \textbf{curve\_fit}, 'f' should be defined like `f(x,*args)`

\end{frame}

\section{Matrix Operation}
\begin{frame}
\frametitle{scipy.linalg}
\begin{itemize}
\item inv(X)
\item solve(A, b)
\item solve\_banded(A, b)
\item det(X)
\item norm(X)
\item pinv(X)
\end{itemize}
\end{frame}

\begin{frame}
\frametitle{scipy.linalg}
\begin{itemize}
\item eig(X)
\item svd(X)
\item suchr(X)
\item qr(X)
\end{itemize}
\end{frame}

\begin{frame}
\frametitle{scipy.sparse.linalg}
\begin{itemize}
\item cg(A, b)
\item gmres(A, b)
\item minres(A, b)
\item eigs(A, k)
\end{itemize}
\end{frame}

\section{Sparse Matrix}
\begin{frame}
\frametitle{scipy.sparse}
\begin{block}{Construction and Basic Operations}
use "csr\_matrix(array\_like)" or "csr\_matrix((data, (row\_ind, col\_ind)) to construct a sparse matrix.

\textbf{Attrs:}

nnz : the number of non-zero elements.

data : the non-zero elements.

indices : the indices of non-zero elements.


\textbf{Methods:}

toarray() : return the array which this sparse matrix represents.



\end{block}
\end{frame}



\section{Graph Algorithms}
\begin{frame}
\frametitle{scipy.sparse.csgraph}
\begin{itemize}
\item connect\_components
\item laplacian
\item shortest\_path
\item dijkstra
\item floyd\_warshall
\item bellman\_ford
\item johnson

\end{itemize}
\end{frame}
\section{Fast Fourier Transform}
\begin{frame}
\frametitle{scipy.fftpack}
\begin{itemize}
\item fft(x)
\item ifft(x)
\item fft2(x)
\item ifft2(x)
\item dct(x)
\item idct(x)
\item dst(x)
\item idst(x)
\end{itemize}
\end{frame}


\section{Image Processing}
\begin{frame}
\frametitle{scipy.ndimage}
\begin{itemize}
\item convolve(input, weight)
\item gaussian\_filter(input, sigma)
\item affine\_transform(input, matrix)
\item rotate(input, angle)
\end{itemize}
\end{frame}

\section{I/O}
\begin{frame}
\frametitle{scipy.io}
\begin{itemize}
\item loadmat(filename)
\item savemat(filename, mdict)
\end{itemize}
\end{frame}

\section{More About SciPy}
\begin{frame}
\frametitle{This outline does not include}
\begin{itemize}
\item scipy.constant
\item scipy.special
\item scipy.spatial
\item scipt.stat
\end{itemize}
Please check the material. qwq
\end{frame}
\end{document}
\
